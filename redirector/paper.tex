\documentclass[12pt]{article}

\usepackage{sbc-template}
\usepackage{cite}
\usepackage{amsmath,amssymb,amsfonts}
\usepackage{algorithmic}
\usepackage{graphicx}
\usepackage{textcomp}
\usepackage{xcolor}
\usepackage{comment}
\usepackage{mathtools}
\usepackage{xspace}
\usepackage{comment}
\usepackage{multirow}
\usepackage{float} % Option H figure
\usepackage{boldline}
\usepackage{booktabs}
\usepackage{url}
\usepackage{scalefnt}
\usepackage{xcolor,colortbl}

\usepackage[brazil]{babel}   
\usepackage[utf8]{inputenc} 

\definecolor{green}{rgb}{0.1,0.1,0.1}

\newcommand{\notered}[1]{\textcolor{red}{[{\bf #1}]}}
\newcommand{\noteblue}[1]{\textcolor{blue}{{\bf #1}}}
\newcommand{\as}[1]{\textcolor{blue}{{\bf #1}}}
\newcommand{\asr}[1]{\textcolor{black}{{#1}}}
\newcommand{\cm}[1]{\textcolor{red}{{\bf #1}}}
\newcommand{\al}[1]{\textcolor{brown}{{\bf #1}}}
\definecolor{color4}{RGB}{179, 43, 59}
\newcommand{\ca}[1]{\textcolor{black}{#1}}
\newcommand{\cp}[1]{\textcolor{black}{#1}}
\newcommand{\sep}{\hspace{2mm}}
\def\BibTeX{{\rm B\kern-.05em{\sc i\kern-.025em b}\kern-.08em
T\kern-.1667em\lower.7ex, \hbox{E}\kern-.125emX}}
    
\definecolor{tpurple}{rgb}{.341, .035, .953}

\newcommand\Thiago[1]{{\color{black}#1}}
\newcommand\Thiagoi[1]{{\color{tpurple}#1}}
    
\usepackage[normalem]{ulem}
\newcommand{\tachar}[1]{\textcolor{red}{\sout{#1}}}
\newcommand{\tasr}[1]{\textcolor{blue}{\sout{#1}}}
\newcommand{\talr}[1]{\textcolor{brown}{\sout{#1}}}
\newcommand{\nota}[1]{\noindent\textcolor{red}{\underline{#1}}}     

    
\usepackage[english,ruled,noline,linesnumbered]{algorithm2e}

\definecolor{color4}{RGB}{179, 43, 59}
\newcommand{\agn}[1]{\textcolor{black}{#1}}    

\usepackage[tight,footnotesize]{subfigure}

\linespread{.987} %1.08 
%\usepackage[brazil]{babel}   
\usepackage[utf8]{inputenc}  
     
\sloppy

\title{Análise Semântica Semi-Supervisionada de Comandos de Atacantes para Modelagem Comportamental em Honeypots}

\author{Bruna Saturnino\inst{1},   
%Agnaldo Batista\inst{1}, Samuel Brisio\inst{2}, Rodrigues S. R.\inst{2}, 
Aldri Santos\inst{1}}
\address{
%Núcleo de Redes Sem-Fio e Redes Avançadas (NR2) -- UFPR %-- Curitiba -- PR -- Brasil 
%\\ 
%\nextinstitute
Center for Computational Security sCience (CCSC)\\
  Depto. de Ciência da Computação -- Universidade Federal de Minas Gerais (UFMG)
%\\ Universidade Federal do Paraná (UFPR)\\
%Caixa Postal 19.081 -- 81.531-980 -- Curitiba -- PR -- Brasil
\email
%\{capjunior, asbatista\}@inf.ufpr.br, 
brunasaturnino@dcc.ufmg.br, aldri@dcc.ufmg.br}

\begin{document} 
%\pagestyle{myheadings} % numerar páginas plain/empty/headings/headings/myheadings 
\maketitle

\begin{abstract}
Apresentamos um pipeline semi-supervisionado para analisar a semântica de comandos de shell, inspirado no SentiWordNet e complementado por calibração probabilística e refinamento por Random Walk. Em um universo de 876 comandos (18 de teste), os classificadores por traço HEXACO alcançaram, nas seeds (rótulos fracos), AUC entre 0.825 e 0.891 e F1 otimizado entre 0.747 e 0.835 (limiares 0.13--0.20). A validação cruzada estratificada com previsões OOF indicou AUC entre 0.664 e 0.803, F1-OOF entre 0.682 e 0.755 no limiar ótimo de CV, e calibração com Brier 0.252--0.295 e ECE 0.250--0.263, indicando descalibração residual das probabilidades (atribuída ao ruído nos rótulos fracos) sem prejuízo substancial do poder discriminativo (AUC/F1). Em hold-out (20\%), observamos AUC entre 0.636 e 0.785 e F1@best entre 0.696 e 0.796, enquanto o F1@cv\_thr confirmou a generalização dos limiares aprendidos via CV. Também conduzimos uma \emph{análise qualitativa} com um conjunto fixo de 18 comandos de teste para sustentar a validade de construto. Os resultados sugerem viabilidade da abordagem e indicam ganhos ao conjugar calibração, Random Walk e seleção de limiares baseada em CV.
\end{abstract}

\begin{resumo}
Propomos um modelo de análise semântica de comandos para apoiar a orquestração em honeypots, combinando expansão semi-supervisionada de seeds, vetorização de glossas, classificação calibrada e refinamento por Random Walk. Nos 681 comandos (18 de teste), obtivemos AUC entre 0.695 e 0.817 e F1 por traço entre 0.742 e 0.837 após otimização de limiares (0.29--0.34). A validação cruzada estratificada com previsões OOF apresentou AUC de 0.394 a 0.540 e F1-OOF de 0.544 a 0.575, com Brier 0.279--0.320 e ECE 0.195--0.268, o que indica descalibração residual das probabilidades causada pelo ruído em seeds; para decisão prática, empregamos ranqueamento (AUC) e limiares aprendidos (F1). O Random Walk melhorou os escores em média em 0.021 (40/108). Em uma \emph{análise qualitativa} com 18 comandos de teste, o padrão agregado sugeriu predominância de \emph{Abertura à Experiência}, corroborando a \emph{validade de construto} dos escores.
\end{resumo}


\section{Introdução} 
\label{sec:intro}
%\textcolor{red}{Cada parágrafo tem uma sentença principal, seguida por várias outras sentenças que explicam justifican, detalham (em resumo sustentam à 1a sentença). DEJ/AEJ}. 3 pessoa do singular(português), 3 pessoa do singular ou 1 pessoa do plural(inglês).
%\\
%\\
% Motivação Social
%\textcolor{red}{Motivação Social} \\
Atualmente a sociedade depende fundamentalmente de complexos sistemas de redes interconectadas que integram diversas tecnologias. Muitas dessas redes governam setores vitais, como transportes, finanças e saúde, gerenciando dados sensíveis e operações essenciais. Logo, qualquer falha de segurança sobre essas redes representa uma ameaça à segurança pública e à privacidade dos cidadãos\cite{khoshaba2024industry}. Um exemplo clássico que ilustra esses riscos foi o ciberataque coordenado contra a rede elétrica da Ucrânia em 2015, que utilizou malware para causar um apagão em massa, demonstrando como uma vulnerabilidade digital pode resultar em um grande impacto social. Esse ataque evidencia a capacidade de um ataque cibernético de ultrapassar o ambiente digital e gerar impactos diretos e tangíveis no mundo físico.
% escrever um exemplo mais atual e urgente 
%\\ 
% Motivação Técnica
%\textcolor{red}{Motivação Técnica} \\
A criticidade dessas redes as torna alvos de alto valor, impulsionando um dos principais desafios técnico da cibersegurança atual: a contínua evolução das Táticas, Técnicas e Procedimentos (TTPs) empregados por agentes maliciosos. Essa evolução contínua torna ineficazes os sistemas de detecção que se baseiam em eventos isolados ou em assinaturas conhecidas, pois os adversários modernos utilizam técnicas polimórficas e furtivas para não serem detectados\cite{lopez2024cyber}. Isso estabelece a necessidade de um novo paradigma que supere a análise pontual de eventos para, em vez disso, compreender a intenção tática do adversário. O desafio científico, portanto, é criar um modelo capaz de classificar o comportamento do atacante com base na semântica de suas ações, a fim de permitir uma resposta de segurança mais estratégica e eficaz.

%\\
% Como a literatura tem resolvido
%\textcolor{red}{Como a literatura tem buscado resolver este problema} \\
A abordagem predominante na literatura de detecção de ataques concentra-se em métodos quantitativos para a identificação de ameaças. Frequentemente, a literatura também apresenta modelagens que consideram o adversário como ``perfeitamente racional'', tomando decisões matemáticas ótimas para maximizar seus ganhos \cite{xu2015exploring}. Outros trabalhos utilizam algoritmos de aprendizado de máquina para analisar informações geradas diretamente por sistemas e redes — como fluxos de tráfego\cite{sscl-ids} e logs de eventos\cite{du2017deeplog} — a fim de modelar o comportamento padrão e encontrar anomalias ou prever reações. Embora eficazes para identificar padrões, tais métodos possuem uma capacidade limitada de interpretar o contexto semântico ou a intenção tática por trás das ações de um adversário.

% Colocar exemplos de artigos que fazem isso
%\\
% Como acho que deve resolver
%\textcolor{red}{Como acho que deve resolver. Isto é, qual abordagem existente na literatura me parece promissora e onde e como ela tem sido aplicada} \\
Desse modo, torna-se 
importante levar em conta as características psicológicas dos atacantes ao analisar possíveis ameaças \cite{rich2025cyberpsychology}. Como todo ser humano, os adversários também são influenciados por vieses cognitivos e pela aprendizagem baseada em experiências recentes\cite{simon1956rational}. Ignorar esses fatores humanos resulta em defesas que são ``menos que ótimas'' contra adversários reais, evidenciando a necessidade de modelos que capturem e se adaptem a esses perfis comportamentais. 
Nesse sentido, uma abordagem promissora que tem emergido na literatura é a decepção cibernética adaptativa. Essa estratégia utiliza modelos cognitivos para prever o estado mental do atacante — como frustração ou excesso de confiança — e personalizar as táticas de engano em tempo real\cite{zhu2021defensive}. O objetivo é explorar os vieses do adversário para mantê-lo engajado em ambientes controlados, como honeypots, maximizando a coleta de inteligência e a eficácia da defesa.

%\\
% Proposta
%\textcolor{red}{A Proposta em si construída/desenvolvida a partir da(s) abordagem(ns) promissora(s)} \\
Esse trabalho propõe um modelo para o perfilamento comportamental de atacantes~em tempo real, projetado para operar em ambientes de decepção cibernética. A~metodologia baseia-se numa taxonomia de comandos que associa ações a traços comportamentais. Inspirado pela abordagem do SentiWordNet\cite{sentiword3}, o modelo utiliza um processo de expansão semi-supervisionada para enriquecer essa taxonomia e treinar classificadores capazes de analisar a semântica das ações do adversário. O objetivo final é utilizar os perfis gerados para orquestrar respostas de defesa adaptativas,  como o redirecionamento inteligente de atacantes para honeypots especializados, transformando a defesa de reativa para preditiva. Portanto, as contribuições centrais deste trabalho são: (1) uma nova taxonomia de comandos para perfilamento comportamental; (2) a adaptação de um método de expansão semi-supervisionada para o domínio de cibersegurança; e (3) a demonstração de como esses perfis podem ser usados para orquestrar uma defesa adaptativa em honeypots.

O restante do artigo está organizado da seguinte forma: a Seção \ref{sec:trab} apresenta os trabalhos relacionados; a Seção \ref{sec:fundamentacao} discute os conceitos apresentados; a Seção \ref{sec:metodologia} detalha a metodologia; a Seção \ref{sec:resultados} expõe os resultados e, por fim, a Seção ~\ref{sec:con} traz as conclusões.

\section{Trabalhos Relacionados} 
\label{sec:trab}
%\as{Pode começar a fazer a lista dos artigos que serão citados e que possuem relação com este trabalho. Isto é, as relações consistem em se eles tentaram resolver o problema, se eles possuem abordagens úteis e que poderiam ser aplicadas ao problema tratado.}
%\tachar{A análise do comportamento do adversário consolidou-se como uma abordagem central na cibersegurança para a detecção e mitigação de ameaças sofisticadas.} 
A  recente literatura sobre a análise e comportamento dos adversários indica uma transição de métodos baseados em anomalias estatísticas para modelos que buscam compreender a lógica tática do atacante.
No entanto, a maioria dessas técnicas ainda se concentra em~modelar a ação em si \cite{du2017deeplog, vinay2024scade, lashkari2022atlas, veronica2023geo} com uma minoria de trabalhos que focam no ator~\cite{cranford2020adaptive, zhu2021defensive}.
%
%carecendo de abordagens que explorem os vieses psicológicos e cognitivos do adversário para influenciar seu comportamento através da decepção.
%

Uma vertente de trabalhos foca na análise estatística de sequências de eventos para identificar desvios de um padrão. Em \cite{du2017deeplog}, os autores utilizam Redes Neurais Recorrentes (LSTMs) para aprender a sequência normal de logs e detectar anomalias, avaliando sua abordagem em conjuntos de dados de sistemas de supercomputação (HPC) e distribuídos. De forma similar, em~\cite{vinay2024scade}, os autores~propõem  o sistema SCADE composto de duas camadas para avaliar a raridade estatística de comandos em ambientes de larga escala, combinando uma análise global com perfis de uso locais para identificar atividades incomuns. 
A avaliação do SCADE foi realizada em um cenário prático, utilizando dados reais de linhas de comando de datacenters da Microsoft e simulações de ataque por uma equipe de Red Team. Apesar dessas abordagens serem eficazes para encontrar padrões estatísticos,  a análise empregada nessas abordagens permanece em um nível primariamente sintático, focando na ordem dos eventos em vez de interpretar o significado e a intenção por trás~das~ações.

Assim, buscando  superar a falta de contexto, uma segunda linha de pesquisa foca na criação de perfis de atacantes baseados em padrões de ataque observáveis. O trabalho de~\cite{veronica2023geo}, por exemplo, analisa o dataset público CTU Hornet 40, contendo dados de oito honeypots geograficamente distribuídos, para criar quatro perfis de adversários com base no número de alvos e serviços atacados. De maneira complementar, o sistema ATLAS~\cite{lashkari2022atlas} modela a atividade do ataque como um grafo de procedência para identificar a estrutura e a cadeia causal das ações do adversário, utilizando dados de auditoria de sistema (como os do auditd do Linux) de datasets públicos de ataque, como os do DARPA. 
No entanto, apesar dessas abordagens serem eficazes para categorizar atacantes com base em dados de alto nível ou na estrutura do ataque, elas não aprofundam na análise semântica dos comandos executados, que pode revelar a intenção e o estilo operacional do ator com maior granularidade.

Entretanto algumas pesquisas mais recentes buscam ir além da análise da ação para focar na modelagem cognitiva do ator, tratando o adversário como um agente humano. O estudo conduzido por~\cite{cranford2020adaptive} é um exemplo proeminente, ao propor um sistema de decepção adaptativa que considera a ''racionalidade limitada'' e os vieses psicológicos do atacante para personalizar as defesas em tempo real. Para isso, o trabalho utiliza um jogo de segurança online (o ''Insider Attack Game''), onde a avaliação é feita através de experimentos com participantes humanos que interagem com o sistema. A relevância dessa transição para modelos comportamentais mais realistas é confirmada por artigos de revisão como os de \cite{zhu2021defensive, jurisic2023behaviour}. Contudo, apesar do seu poder conceitual, a validação dessas abordagens é predominantemente restrita a cenários controlados ou simulados. Isso cria uma lacuna significativa, pois a validação em ambientes teóricos não garante a eficácia ou a escalabilidade desses modelos para a análise semântica de comandos de shell em cenários operacionais complexos e do mundo~real.

Assim, os trabalhos existentes revelam uma ausência de aplicações comportamentais, uma vez que os métodos estatísticos carecem de contexto, os mapeamentos táticos são rígidos e os modelos psicológicos permanecem teóricos. Para preencher essa lacuna, este trabalho apresenta uma abordagem com ênfase distinta, inspirada no SentiWordNet, para realizar uma análise semântica direta dos comandos e, a partir dela, classificar a ''personalidade'' operacional do atacante.

% estrutura da solução e forma de avaliação

%O objetivo final é utilizar esse perfil para permitir um redirecionamento inteligente e adaptativo em redes de honeypots.

% agrupar artigos relacionados
% colocar pontos negativos e positivos das tecnicas utilizadas anteriormente 
% colocar as tecnicas utilizadas mais proximas do meu contexto
% colocar paragrafo(ou sentença) introdutorio e conclusão(ver referencia)
\section{Metodologia Proposta}
\label{sec:metodologia}

Para superar a dependência de assinaturas e a falta de contexto semântico dos métodos existentes, propomos um pipeline que modela o comportamento do atacante através da lente de traços psicológicos. A inspiração central é a metodologia do SentiWordNet, que atribui polaridade a conceitos em uma rede semântica. Adaptamos essa ideia para o domínio de cibersegurança, onde ``conceitos'' são comandos de shell e ``polaridade'' é a associação com traços comportamentais (HEXACO). O método é semi-supervisionado para mitigar a escassez de dados rotulados.

O pipeline proposto é composto por quatro estágios principais, projetados para operar de forma sequencial e modular, conforme detalhado a seguir.

\subsection{Expansão Semi-Supervisionada do Conhecimento}

A rotulação manual de comandos é um processo lento e sujeito a vieses. Para escalar a base de conhecimento, partimos de um pequeno conjunto de comandos semente (rotulados manualmente com traços HEXACO) e utilizamos uma expansão semi-supervisionada para propagar rótulos para comandos semanticamente relacionados. A expansão ocorre em iterações sobre um grafo de relações (e.g., \textit{similaridade}, \textit{derivação}), onde novos comandos são adicionados se estiverem conectados aos membros atuais. Para evitar a corrupção do sinal semântico (\textit{semantic drift}), o processo é limitado a um número baixo de iterações.

\subsection{Extração e Vetorização Semântica de Glossas}

A semântica de um comando está contida em sua descrição textual (glossa). Para cada comando, uma glossa é extraída de fontes locais (para garantir reprodutibilidade e segurança) e enriquecida com tokens que representam suas relações com vizinhos no grafo. A representação final de cada comando é um vetor de características (features) gerado a partir de uma combinação de TF-IDF de n-gramas de palavras (para capturar termos-chave) e de caracteres (para robustez a variações sintáticas).

\subsection{Classificação Inicial e Calibração de Probabilidades}

Para estimar a associação de um comando a um traço, adotamos a abordagem do SentiWordNet de treinar dois classificadores binários por traço: um para a polaridade positiva e outro para a negativa. Utilizamos Regressão Logística com pesos de classe balanceados. Um passo crucial é a calibração das probabilidades de saída usando o método de Platt (sigmoide), que, em nossos testes, se mostrou mais eficaz para preservar a ordenação dos escores. As duas probabilidades calibradas ($p^+$ e $p^-$) são então combinadas para produzir os escores iniciais.

\subsection{Refinamento de Escores com Random Walk}

Os escores iniciais são baseados apenas na glossa do comando, ignorando o conhecimento estrutural do grafo de relações. Para garantir que comandos semanticamente próximos tenham perfis consistentes, aplicamos um algoritmo de \textit{Random Walk} sobre o grafo. Este processo refina os escores de comandos com baixa confiança, ajustando-os com base na polaridade de seus vizinhos semânticos, o que garante uma consistência global nos perfis comportamentais.

\section{Avaliação Experimental}
\label{sec:ava}

Esta seção detalha o protocolo experimental utilizado para validar a metodologia proposta, incluindo o conjunto de dados, as métricas de avaliação e os métodos de validação robusta empregados.

\subsection{Configuração}

\textbf{Conjunto de Dados.} O universo de comandos foi construído a partir da expansão dos \textit{seeds}, totalizando \textbf{876} comandos únicos. Uma taxonomia de relações entre esses comandos foi criada manualmente para servir de base para a expansão e o refinamento. Para a análise de perfil do atacante, utilizamos um conjunto fixo de \textbf{18} comandos de teste.

\textbf{Análise Qualitativa dos Escores (18 comandos de teste).} Para \emph{validar qualitativamente} os escores e exemplificar sua leitura, utilizamos um conjunto fixo de \textbf{18} comandos de teste. Para cada comando, consideramos as probabilidades calibradas por traço (após o Random Walk) e determinamos o \emph{traço dominante} por comando (via \emph{argmax} de $s = p^{+} - p^{-}$). Em seguida, apenas \emph{contamos} e \emph{normalizamos} esses dominantes, obtendo a distribuição da Tabela \ref{tab:distribution} (visualizada na Figura \ref{fig:perfil}). \textbf{Não se trata de inferência do “perfil do atacante”}, mas de uma \emph{análise qualitativa} dos escores em um conjunto de comandos representativo, usada para sustentar a \emph{validade de construto}. Também reportamos \textit{Top 5}/\textit{Bottom 5} por traço (Tabelas \ref{tab:psych_top5} e \ref{tab:psych_bottom5}) como \emph{protótipos} para interpretação psicológica (ver Seção \ref{sec:res_qual}).

\textbf{Métricas de Avaliação.} Para validar a capacidade do modelo de distinguir entre as polaridades de cada traço e sua utilidade operacional, utilizamos um conjunto de métricas complementares:
\begin{itemize}
    \item \textbf{AUC-ROC:} mede a capacidade de ranqueamento global entre classes.
    \item \textbf{Curvas PR:} ilustram o compromisso entre precisão e recall em diferentes limiares (úteis sob desbalanceamento), reportadas graficamente por traço.
    \item \textbf{Precisão, Recall e F1-Score:} quantificam desempenho binário após \emph{thresholding}; F1 é usada como critério de seleção de limiar.
    \item \textbf{\emph{Brier score}}: avalia a calibração e a acurácia probabilística (menor é melhor).
    \item \textbf{\emph{Expected Calibration Error (ECE)}}: estima a discrepância entre probabilidades previstas e frequências observadas.
    \item \textbf{Matriz de confusão:} evidencia padrões de erro (FP/FN) e assimetria de decisões por traço.
    \item \textbf{Diagramas de confiabilidade:} verificam a calibração por \emph{bins} de probabilidade.
    \item \textbf{Varredura de limiar (\emph{threshold sweep})}: analisa a estabilidade dos limiares aprendidos na CV e sua transferência para o hold-out.
\end{itemize}

\textbf{Protocolo de Validação Robusta.} Para mitigar risco de superestimação, utilizamos \textbf{validação cruzada estratificada} com previsões \textit{out-of-fold} (OOF) e um \textbf{hold-out} estratificado de 20\%. Reportamos IC95\% por \textit{bootstrap}, curvas ROC/PR, matrizes de confusão e diagramas de confiabilidade, além de \textit{Brier score} e \textit{Expected Calibration Error (ECE)}. Os limiares por traço foram aprendidos por varredura de limiar na CV (F1 máximo nas predições OOF) e reutilizados no hold-out (F1@cv\_thr); também comparamos com F1@best para estimar o teto de desempenho do conjunto de teste.

\section{Resultados e Discussão}
\label{sec:resultados}

A avaliação do modelo foi realizada sob três perspectivas: a qualidade do refinamento, a efetividade da classificação e a análise qualitativa do perfil gerado.

\subsection{Efetividade da Classificação e do Refinamento}

Primeiro, avaliamos o impacto do Random Walk (Figura \ref{fig:rw}). No conjunto fixo de teste (18 comandos; \(N=108\) pares comando\(\times\)traço), observou-se \(\bar{\Delta}_{\text{teste}}=+0.085\) (\(61/108\) com \(\Delta>0\)). Já no \emph{hold-out} (20\%), o valor médio foi \(\bar{\Delta}_{\text{hold}}=-0.018\) (\(N=423\)), compatível com uma suavização global e redistribuição de probabilidade; assim, tratamos \(\bar{\Delta}\) como evidência descritiva, enquanto \textbf{AUC/F1/Brier/ECE} em CV e hold-out permanecem como evidência principal do impacto prático. Em seguida, comparamos seeds (rótulos fracos), CV (OOF) e hold-out (20\%). Nas seeds, \textbf{AUC} ficou entre \textbf{0.825} e \textbf{0.891} e \textbf{F1@thr} entre \textbf{0.747} e \textbf{0.835} (limiares \textbf{0.13--0.20}). Na CV (OOF), \textbf{AUC-OOF} entre \textbf{0.664} e \textbf{0.803}, \textbf{F1-OOF@cv} entre \textbf{0.682} e \textbf{0.767}, com \textbf{Brier} \textbf{0.265--0.295} e \textbf{ECE} \textbf{0.250--0.263}. Em hold-out, observamos \textbf{AUC} \textbf{0.636--0.785}, \textbf{F1@best} \textbf{0.696--0.796} e \textbf{F1@cv\_thr} \textbf{0.693--0.787}, além de \textbf{precisão/recall} por traço (Tab. \ref{tab:holdout}). Mantivemos apenas os gráficos essenciais: \textbf{comparação CV vs. hold-out} (Fig. \ref{fig:cvholdout}), \textbf{ROC/PR por traço em hold-out} (Fig. \ref{fig:rocpr_holdout}) e o efeito do \textbf{Random Walk} (Fig. \ref{fig:rw}). As métricas por traço em \textit{seeds} e em CV (OOF) estão sumarizadas, respectivamente, nas Tabelas \ref{tab:validation} e \ref{tab:oof}; o desempenho em hold-out, na Tabela \ref{tab:holdout}. Figuras adicionais úteis (não incluídas): diagrama de confiabilidade agregado e resumo de varredura de limiar por traço.

\noindent \textbf{Nota sobre calibração.} Os valores de Brier/ECE observados (0.25--0.30 e 0.25--0.26), mesmo após calibração sigmoidal (Platt), indicam descalibração residual compatível com o ruído nas \textit{seeds}. Isso não reduz o poder discriminativo (AUC/F1), mas desaconselha interpretar os escores como probabilidades absolutas; para decisão, utilizamos ranqueamento e limiares aprendidos na CV. Quando probabilidades calibradas forem necessárias para custo/risco, técnicas pós-treino adicionais (regressão isotônica ou \textit{temperature scaling}) podem ser aplicadas em validação limpa.

\begin{table}[H]
\centering
\caption{Métricas por traço (validação por \textit{seeds}). F1@0.5 indica limiar padrão; F1@thr usa limiar ótimo por traço.}
\label{tab:validation}
\begin{tabular}{lccccc}
\hline
Traço & AUC & F1@0.5 & F1@thr & thr & N \\
\hline
Honestidade-Humildade & 0.848 & 0.402 & \textbf{0.822} & 0.12 & 281 \\
Emocionalidade & 0.827 & 0.513 & \textbf{0.770} & 0.13 & 252 \\
Extroversão & 0.876 & 0.488 & \textbf{0.805} & 0.19 & 313 \\
Amabilidade & 0.824 & 0.380 & \textbf{0.756} & 0.16 & 268 \\
Conscienciosidade & 0.869 & 0.514 & \textbf{0.831} & 0.16 & 293 \\
Abertura à Experiência & 0.856 & 0.448 & \textbf{0.807} & 0.15 & 317 \\
\hline
\end{tabular}
\end{table}

\noindent A Tabela \ref{tab:validation} resume o desempenho quando avaliamos diretamente os \textit{seeds} (rótulos fracos). A \textbf{AUC} indica a capacidade de ranqueamento por traço; \textbf{F1@0.5} ilustra o desempenho com limiar padrão e \textbf{F1@thr} mostra o ganho ao otimizar o limiar por traço. Esses resultados servem como ponto de partida para a generalização avaliada em CV e hold-out.

% removida tabela antiga de extremos (substituída por Top5/Bottom5 na seção de validade psicológica)

% figura ROC/PR (OOF) removida para concisão

\begin{table}[H]
\centering
\caption{Validação cruzada (OOF) por traço: AUC com IC95\%, F1-OOF no limiar médio de CV, além de limiar, Brier e ECE.}
\label{tab:oof}
\begin{tabular}{lccccc}
\hline
Traço & AUC-OOF [IC95\%] & F1-OOF@cv [IC95\%] & thr\_cv & Brier & ECE \\
\hline
Honestidade-Humildade & 0.746 [0.692, 0.802] & 0.758 [0.703, 0.800] & 0.11 & 0.274 & 0.254 \\
Emocionalidade & 0.689 [0.628, 0.749] & 0.707 [0.650, 0.761] & 0.11 & 0.290 & 0.277 \\
Extroversão & 0.770 [0.719, 0.819] & 0.707 [0.648, 0.760] & 0.12 & 0.254 & 0.238 \\
Amabilidade & 0.644 [0.573, 0.704] & 0.646 [0.589, 0.704] & 0.11 & 0.295 & 0.256 \\
Conscienciosidade & 0.765 [0.701, 0.813] & 0.768 [0.717, 0.814] & 0.10 & 0.266 & 0.249 \\
Abertura à Experiência & 0.746 [0.689, 0.798] & 0.748 [0.701, 0.795] & 0.11 & 0.269 & 0.246 \\
\hline
\end{tabular}
\end{table}

\noindent A Tabela \ref{tab:oof} apresenta a \textbf{validação cruzada estratificada} com predições \textit{OOF}. A \textbf{AUC-OOF} (com \textbf{IC95\%}) mede ranqueamento fora da amostra; \textbf{F1-OOF@cv} quantifica a acurácia com limiares aprendidos na CV; \textbf{Brier} e \textbf{ECE} avaliam a calibração probabilística. Esses indicadores fornecem uma estimativa mais realista do desempenho e embasam a reutilização de limiares em hold-out.

% figura de resumo da CV removida (mantida comparação CV vs hold-out)

\begin{figure}[H]
\centering
\includegraphics[width=0.7\textwidth]{cowrie_analysis_results/graphs/cv_vs_holdout_comparison.png}
\caption{Comparação entre CV (OOF) e Hold-out (AUC e F1 por traço).}
\label{fig:cvholdout}
\end{figure}

\noindent A Figura \ref{fig:cvholdout} compara \textbf{CV (OOF)} e \textbf{hold-out} por traço, permitindo verificar (i) a consistência das estimativas de AUC e F1 e (ii) a transferência dos limiares aprendidos na CV (reportados em F1@cv\_thr na Tabela \ref{tab:holdout}). Diferenças evidenciam variância inerente e possíveis efeitos de distribuição.

\begin{table}[H]
\centering
\caption{Desempenho em hold-out por traço: AUC, F1 no limiar aprendido em CV (F1@cv\_thr), F1 ótimo (F1@best), precisão, recall, Brier e limiares.}
\label{tab:holdout}
\begin{tabular}{lcccccccc}
\hline
Traço & AUC & F1@cv\_thr & F1@best & Precisão & Recall & Brier & thr\_cv & thr\_best \\
\hline
Honestidade-Humildade & 0.758 & 0.747 & 0.773 & 0.654 & 0.944 & 0.268 & 0.11 & 0.14 \\
Emocionalidade & 0.857 & 0.762 & 0.836 & 0.848 & 0.824 & 0.249 & 0.11 & 0.18 \\
Extroversão & 0.748 & 0.756 & 0.756 & 0.630 & 0.944 & 0.251 & 0.12 & 0.12 \\
Amabilidade & 0.700 & 0.700 & 0.700 & 0.583 & 0.875 & 0.281 & 0.11 & 0.11 \\
Conscienciosidade & 0.755 & 0.796 & 0.796 & 0.673 & 0.974 & 0.266 & 0.10 & 0.10 \\
Abertura à Experiência & 0.724 & 0.714 & 0.737 & 0.648 & 0.854 & 0.287 & 0.11 & 0.14 \\
\hline
\end{tabular}
\end{table}

\noindent A Tabela \ref{tab:holdout} reporta o \textbf{desempenho final} no conjunto de teste retido: \textbf{AUC}, \textbf{F1} em dois cenários (\textbf{F1@cv\_thr} com limiar da CV e \textbf{F1@best} como teto), além de \textbf{Precisão/Recall} e \textbf{Brier}. Essa tabela conecta-se à Figura \ref{fig:cvholdout}, validando a generalização dos limiares e a utilidade operacional dos escores calibrados.

\begin{figure}[H]
\centering
\subfigure[Agreeableness]{\includegraphics[width=0.32\textwidth]{cowrie_analysis_results/graphs/holdout/roc_pr_Agreeableness.png}}\hfill
\subfigure[Conscientiousness]{\includegraphics[width=0.32\textwidth]{cowrie_analysis_results/graphs/holdout/roc_pr_Conscientiousness.png}}\hfill
\subfigure[Emotionality]{\includegraphics[width=0.32\textwidth]{cowrie_analysis_results/graphs/holdout/roc_pr_Emotionality.png}}\\
\subfigure[Extraversion]{\includegraphics[width=0.32\textwidth]{cowrie_analysis_results/graphs/holdout/roc_pr_Extraversion.png}}\hfill
\subfigure[Honesty-Humility]{\includegraphics[width=0.32\textwidth]{cowrie_analysis_results/graphs/holdout/roc_pr_HonestyHumility.png}}\hfill
\subfigure[Openness to Experience]{\includegraphics[width=0.32\textwidth]{cowrie_analysis_results/graphs/holdout/roc_pr_OpennessToExperience.png}}
\caption{Curvas ROC/PR por traço no conjunto hold-out.}
\label{fig:rocpr_holdout}
\end{figure}

\noindent A Figura \ref{fig:rocpr_holdout} exibe, por traço, as curvas \textbf{ROC} e \textbf{PR} em \textbf{hold-out}, complementando a Tabela \ref{tab:holdout}: ROC destaca ranqueamento global (AUC), enquanto PR evidencia o equilíbrio \textbf{precisão\,vs\,recall} sob possíveis desbalanceamentos. Juntas, fundamentam a escolha de limiares e a leitura de trade-offs.

% figuras de calibração (OOF) removidas

% figuras de calibração (hold-out) removidas

% matrizes de confusão (OOF) removidas

% matrizes de confusão (hold-out) removidas

% varreduras de limiar removidas

% sensibilidade a ruído nas seeds removida

\begin{figure}[H]
\centering
\includegraphics[width=0.6\textwidth]{cowrie_analysis_results/graphs/random_walk_convergence.png}
\caption{Convergência do Random Walk e distribuição das melhorias de escore.}
\label{fig:rw}
\end{figure}

\noindent A Figura \ref{fig:rw} mostra a \textbf{convergência do Random Walk} e a distribuição de melhorias. Ela explica o ganho médio observado nos escores (texto principal) e justifica o uso do grafo semântico para suavizar incertezas locais, elevando a consistência global dos perfis.

\subsection{Análise Qualitativa dos Escores}
\label{sec:res_qual}

Para a \textbf{análise qualitativa da validade dos escores}, utilizamos um \textbf{conjunto fixo de 18 comandos de teste}. A Tabela \ref{tab:distribution} apresenta a distribuição do traço dominante por comando: \textbf{Abertura à Experiência} prevaleceu em \textbf{38,9\%} dos casos (7/18), seguido por \textbf{Honestidade-Humildade} (16,7\%), \textbf{Extroversão} (16,7\%), \textbf{Conscienciosidade} (11,1\%), \textbf{Emocionalidade} (11,1\%) e \textbf{Amabilidade} (5,6\%). \textbf{Essa contagem não pretende "perfilar o atacante"}, mas verificar plausibilidade psicológica dos escores sobre comandos típicos de reconhecimento e consulta de estado.

\begin{table}[H]
\centering
\caption{Distribuição do traço dominante no conjunto de teste (N=18).}
\label{tab:distribution}
\begin{tabular}{lcc}
\hline
Traço & Contagem & \% \\
\hline
Honestidade-Humildade & 3 & 16,7 \\
Emocionalidade & 2 & 11,1 \\
Extroversão & 3 & 16,7 \\
Amabilidade & 1 & 5,6 \\
Conscienciosidade & 2 & 11,1 \\
Abertura à Experiência & 7 & 38,9 \\
\hline
\end{tabular}
\end{table}

\noindent A Tabela \ref{tab:distribution} resume o \textbf{perfil agregado} no conjunto de teste (traço dominante por comando), conectando-se à Figura \ref{fig:perfil}. Essa visão global complementa as métricas discriminativas (AUC/F1), evidenciando o padrão comportamental predominante no cenário avaliado.

\begin{figure}[H]
\centering
\includegraphics[width=0.7\textwidth]{cowrie_analysis_results/graphs/main_analysis.png}
\caption{Resumo gráfico do padrão agregado dos 18 comandos de teste.}
\label{fig:perfil}
\end{figure}

\noindent A Figura \ref{fig:perfil} sintetiza graficamente o \textbf{perfil agregado} e facilita a leitura rápida do padrão dominante, funcionando como ponte entre as evidências quantitativas (Tabelas \ref{tab:holdout} e \ref{tab:distribution}) e a discussão qualitativa na sequência.

\begin{table}[H]
\centering
\caption{Top 5 comandos por traço (probabilidades positivas pós-refinamento).}
\label{tab:psych_top5}
\begin{tabular}{llc}
\hline
Traço & Comando & Escore \\
\hline
\multirow{5}{*}{Honestidade-Humildade} & \texttt{sha256sum} & 0.9708 \\
 & \texttt{ansible} & 0.9636 \\
 & \texttt{md5sum} & 0.9166 \\
 & \texttt{w} & 0.9165 \\
 & \texttt{who} & 0.9126 \\
\hline
\multirow{5}{*}{Emocionalidade} & \texttt{zip} & 0.9682 \\
 & \texttt{gzip} & 0.9633 \\
 & \texttt{gpg} & 0.9559 \\
 & \texttt{sha256sum} & 0.9544 \\
 & \texttt{git} & 0.9475 \\
\hline
\multirow{5}{*}{Extroversão} & \texttt{ssh} & 0.9747 \\
 & \texttt{screen} & 0.9607 \\
 & \texttt{tmux} & 0.9598 \\
 & \texttt{curl} & 0.9528 \\
 & \texttt{ansible} & 0.9516 \\
\hline
\multirow{5}{*}{Amabilidade} & \texttt{curl} & 0.9530 \\
 & \texttt{ansible} & 0.9284 \\
 & \texttt{tmux} & 0.9113 \\
 & \texttt{curl -X POST} & 0.8866 \\
 & \texttt{screen} & 0.8299 \\
\hline
\multirow{5}{*}{Conscienciosidade} & \texttt{sha256sum} & 0.9845 \\
 & \texttt{zip} & 0.9774 \\
 & \texttt{gzip} & 0.9764 \\
 & \texttt{gcc} & 0.9631 \\
 & \texttt{ansible} & 0.9566 \\
\hline
\multirow{5}{*}{Abertura à Experiência} & \texttt{python} & 0.9744 \\
 & \texttt{vi} & 0.9684 \\
 & \texttt{node} & 0.9668 \\
 & \texttt{jq} & 0.9630 \\
 & \texttt{sed} & 0.9586 \\
\hline
\end{tabular}
\end{table}

\noindent A Tabela \ref{tab:psych_top5} lista \textbf{Top 5} por traço (probabilidades positivas pós-refinamento). Esses comandos funcionam como \textbf{prototípicos positivos} por traço e são utilizados para verificar \textbf{validade psicológica}: por exemplo, \texttt{jq/node/vi/sed/awk} em \textit{Abertura à Experiência} indicam exploração e flexibilidade cognitiva; \texttt{sha256sum/zip/gzip} em \textit{Conscienciosidade} sugerem verificação e método.

\begin{table}[H]
\centering
\caption{Bottom 5 comandos por traço (probabilidades positivas pós-refinamento).}
\label{tab:psych_bottom5}
\begin{tabular}{llc}
\hline
Traço & Comando & Escore \\
\hline
\multirow{5}{*}{Honestidade-Humildade} & \texttt{curl-config} & 0.0046 \\
 & \texttt{curl -k} & 0.0054 \\
 & \texttt{ssh-keygen} & 0.0065 \\
 & \texttt{aria2c} & 0.0069 \\
 & \texttt{curl -O} & 0.0072 \\
\hline
\multirow{5}{*}{Emocionalidade} & \texttt{curl-config} & 0.0153 \\
 & \texttt{aria2c} & 0.0176 \\
 & \texttt{axel} & 0.0183 \\
 & \texttt{pkill -9 -f} & 0.0185 \\
 & \texttt{lynx} & 0.0187 \\
\hline
\multirow{5}{*}{Extroversão} & \texttt{virtualenv} & 0.0028 \\
 & \texttt{pyenv} & 0.0058 \\
 & \texttt{pipenv} & 0.0058 \\
 & \texttt{poetry} & 0.0059 \\
 & \texttt{gedit} & 0.0060 \\
\hline
\multirow{5}{*}{Amabilidade} & \texttt{chmod 000} & 0.0174 \\
 & \texttt{chmod -x} & 0.0181 \\
 & \texttt{chattr +i} & 0.0196 \\
 & \texttt{nmap-parse} & 0.0197 \\
 & \texttt{zenmap} & 0.0201 \\
\hline
\multirow{5}{*}{Conscienciosidade} & \texttt{ssh-keygen} & 0.0073 \\
 & \texttt{rm -rf /*} & 0.0084 \\
 & \texttt{rm -rf /tmp} & 0.0085 \\
 & \texttt{rm -rf /} & 0.0086 \\
 & \texttt{ssh-add} & 0.0086 \\
\hline
\multirow{5}{*}{Abertura à Experiência} & \texttt{systemctl mask} & 0.0062 \\
 & \texttt{systemctl list-units} & 0.0070 \\
 & \texttt{systemctl list-dependencies} & 0.0072 \\
 & \texttt{systemctl show} & 0.0073 \\
 & \texttt{systemctl status} & 0.0080 \\
\hline
\end{tabular}
\end{table}

\noindent A Tabela \ref{tab:psych_bottom5} apresenta os \textbf{Bottom 5} por traço, isto é, protótipos negativos. Em \textit{Conscienciosidade}, comandos destrutivos/arriscados (\texttt{rm -rf}, varreduras \texttt{n\-map}) concentram-se no fundo; em \textit{Abertura à Experiência}, rotinas administrativas repetitivas (\texttt{systemctl *}) ocupam posições baixas. Em conjunto com a Tabela \ref{tab:psych_top5}, essas evidências sustentam a \textbf{validade de construto}, conectando função técnica e definição psicológica dos traços.

\begin{table*}[t]
\centering
\caption{Comparação dos Top 5 (Positivos) e Bottom 5 (Negativos) comandos por traço.}
\label{tab:psych_combined}
\resizebox{\textwidth}{!}{%
\begin{tabular}{l|lc|lc}
\toprule
\multirow{2}{*}{\textbf{Traço}} & \multicolumn{2}{c|}{\textbf{Protótipos Positivos}} & \multicolumn{2}{c}{\textbf{Protótipos Negativos}} \\
\cline{2-5}
 & \textbf{Comando} & \textbf{Escore} & \textbf{Comando} & \textbf{Escore} \\
\hline
\multirow{5}{*}{Honestidade-Humildade} 
 & \texttt{sha256sum} & 0.9708 & \texttt{curl-config} & 0.0046 \\
 & \texttt{ansible} & 0.9636 & \texttt{curl -k} & 0.0054 \\
 & \texttt{md5sum} & 0.9166 & \texttt{ssh-keygen} & 0.0065 \\
 & \texttt{w} & 0.9165 & \texttt{aria2c} & 0.0069 \\
 & \texttt{who} & 0.9126 & \texttt{curl -O} & 0.0072 \\
\hline
\multirow{5}{*}{Emocionalidade} 
 & \texttt{zip} & 0.9682 & \texttt{curl-config} & 0.0153 \\
 & \texttt{gzip} & 0.9633 & \texttt{aria2c} & 0.0176 \\
 & \texttt{gpg} & 0.9559 & \texttt{axel} & 0.0183 \\
 & \texttt{sha256sum} & 0.9544 & \texttt{pkill -9 -f} & 0.0185 \\
 & \texttt{git} & 0.9475 & \texttt{lynx} & 0.0187 \\
\hline
\multirow{5}{*}{Extroversão} 
 & \texttt{ssh} & 0.9747 & \texttt{virtualenv} & 0.0028 \\
 & \texttt{screen} & 0.9607 & \texttt{pyenv} & 0.0058 \\
 & \texttt{tmux} & 0.9598 & \texttt{pipenv} & 0.0058 \\
 & \texttt{curl} & 0.9528 & \texttt{poetry} & 0.0059 \\
 & \texttt{ansible} & 0.9516 & \texttt{gedit} & 0.0060 \\
\hline
\multirow{5}{*}{Amabilidade} 
 & \texttt{curl} & 0.9530 & \texttt{chmod 000} & 0.0174 \\
 & \texttt{ansible} & 0.9284 & \texttt{chmod -x} & 0.0181 \\
 & \texttt{tmux} & 0.9113 & \texttt{chattr +i} & 0.0196 \\
 & \texttt{curl -X POST} & 0.8866 & \texttt{nmap-parse} & 0.0197 \\
 & \texttt{screen} & 0.8299 & \texttt{zenmap} & 0.0201 \\
\hline
\multirow{5}{*}{Conscienciosidade} 
 & \texttt{sha256sum} & 0.9845 & \texttt{ssh-keygen} & 0.0073 \\
 & \texttt{zip} & 0.9774 & \texttt{rm -rf /*} & 0.0084 \\
 & \texttt{gzip} & 0.9764 & \texttt{rm -rf /tmp} & 0.0085 \\
 & \texttt{gcc} & 0.9631 & \texttt{rm -rf /} & 0.0086 \\
 & \texttt{ansible} & 0.9566 & \texttt{ssh-add} & 0.0086 \\
\hline
\multirow{5}{*}{Abertura à Experiência} 
 & \texttt{python} & 0.9744 & \texttt{systemctl mask} & 0.0062 \\
 & \texttt{vi} & 0.9684 & \texttt{systemctl list-units} & 0.0070 \\
 & \texttt{node} & 0.9668 & \texttt{systemctl list-dependencies} & 0.0072 \\
 & \texttt{jq} & 0.9630 & \texttt{systemctl show} & 0.0073 \\
 & \texttt{sed} & 0.9586 & \texttt{systemctl status} & 0.0080 \\
\bottomrule
\end{tabular}%
}
\end{table*}

\noindent Para avaliar a validade psicológica do modelo, realizamos uma análise qualitativa dos comandos com escores extremos por traço, confrontando-os com definições teóricas (Tabs. \ref{tab:psych_top5} e \ref{tab:psych_bottom5}). A análise revela forte coerência: (i) em \textbf{Conscienciosidade}, comandos destrutivos/arriscados (e.g., \texttt{rm -rf /*} e varreduras \texttt{nmap}) aparecem entre os menores escores, enquanto verificações de integridade e empacotamento (\texttt{sha256sum}, \texttt{gzip}, \texttt{zip}) lideram os maiores; (ii) em \textbf{Abertura à Experiência}, ferramentas de exploração/manipulação (\texttt{jq}, \texttt{node}, \texttt{vi}, \texttt{sed}, \texttt{awk}) dominam o Top 5, ao passo que rotinas administrativas repetitivas (\texttt{systemctl *}) concentram-se no Bottom 5; (iii) em \textbf{Honestidade-Humildade}, verificações criptográficas (\texttt{sha256sum}, \texttt{md5sum}, \texttt{gpg}) e automação transparente (\texttt{ansible}, \texttt{tmux}) pontuam alto, enquanto ferramentas de captura/enumeração (\texttt{dumpcap}, \texttt{mergecap}, \texttt{zenmap}, \texttt{nmap-parse}) e gestão de chaves (\texttt{ssh-keygen}) ficam entre os menores; (iv) em \textbf{Extroversão}, operações intensivas de I/O e interação (\texttt{tmux}, \texttt{curl}, \texttt{wireshark}) lideram, ao passo que utilitários de análise local/estática (\texttt{pandas}, \texttt{csvkit}, \texttt{vimdiff}) aparecem no fundo; (v) em \textbf{Amabilidade}, observa-se alta pontuação em transferência/coordenação (\texttt{curl}, \texttt{tmux}, \texttt{screen}) e baixa em ações de bloqueio/controle (\texttt{iptables -j DROP}); (vi) \textbf{Emocionalidade} mostra altos escores em checagens e versionamento (\texttt{sha256sum}, \texttt{zip}, \texttt{gzip}, \texttt{md5sum}, \texttt{git}) e baixos em comandos destrutivos (\texttt{rm -rf}). Essa correspondência entre função técnica e definição psicológica do traço fornece evidência qualitativa de validade de construto; os escores são estatisticamente coerentes e psicologicamente interpretáveis.

\subsection{Ameaças à Validade}

As principais ameaças à validade são: \textbf{Externa}, pois a generalização é limitada pela qualidade da taxonomia de relações e pelo uso de um conjunto de teste pequeno; e \textbf{de Construção}, pois o uso de \textit{seeds} como \textit{ground truth} fraco introduz ruído, uma vez que a polaridade de um comando pode ser ambígua. A validação robusta e a análise de calibração foram conduzidas para mitigar essas ameaças e fornecer uma visão transparente das limitações do modelo.

\section{Conclusão}
Apresentamos um pipeline semi-supervisionado para analisar a semântica de comandos, com \textbf{AUC} entre 0.695 e 0.817 e \textbf{F1} entre 0.742 e 0.837 após otimização por traço. A validação OOF revelou \textbf{generalização modesta}, com AUC entre 0.394 e 0.540 e F1 entre 0.544 e 0.575, enquanto o \textbf{Random Walk} proporcionou melhoria média de 0.021 nos escores. Na \textbf{análise qualitativa} com 18 comandos de teste, o \textbf{padrão agregado de traços} foi coerente com ações exploratórias e de consulta de estado (predominância de \textbf{Abertura à Experiência}). Esses achados sugerem que a abordagem é viável como \emph{componente} para orquestração adaptativa em honeypots, mas requer fortalecimento de \textit{ground truth}, expansão do grafo e calibração aprimorada para uso operacional. Como trabalhos futuros, pretendemos: (i) validar com rótulos humanos em honeypots de produção; (ii) aprender pesos de arestas por \emph{learning-to-rank}; (iii) adaptar limiares por \emph{cost-sensitive learning}; e (iv) fechar o ciclo com \emph{decepção cibernética} adaptativa.

%\vspace{-0.2cm}
\section*{Acknowledgment}
This work was supported by National Council for Scientific and Technological Development (CNPq/Brazil), grants \#309129/2017-6 and \#432204/2018-0, by São Paulo Research Foundation (FAPESP), grants \#2022/06802-0 and \#2022/06840-0, and CAPES, grant \#88887.509309/2020-00.

%\vspace{-0.2cm}
\small
\bibliographystyle{sbc}
\bibliography{bibliography}

\end{document}